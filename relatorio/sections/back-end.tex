\section{Back-end}

\subsection{server.js}
Depois de criar o arquivo .env, que contém as variáveis de ambiente necessárias para o funcionamento 
da aplicação, vamos criar o arquivo principal server.js.

\begin{lstlisting}
    const express = require('express');
    const cors = require('cors');
    const mongoose = require('mongoose');
    require('dotenv').config();
\end{lstlisting}

- express: Framework para criar e gerenciar o servidor HTTP.

- cors: Middleware para habilitar o CORS (Cross-Origin Resource Sharing), 
permitindo que o backend aceite requisições de diferentes origens.

- mongoose: Biblioteca para trabalhar com MongoDB em JavaScript, oferecendo um modelo orientado a objetos.

- dotenv: Carrega as variáveis do arquivo .env para process.env.

\begin{lstlisting}
    const authRoutes = require('./routes/auth');
    const lojaRoutes = require('./routes/loja');
    const userRoutes = require('./routes/user');
\end{lstlisting}

Aqui, são importados arquivos de rotas localizados na pasta routes. 
Essas rotas gerenciam diferentes funcionalidades:

- authRoutes: Gerencia autenticação, como login e registro.

- lojaRoutes: Gerencia operações relacionadas às lojas.

- userRoutes: Gerencia operações relacionadas aos usuários.

\vspace{1cm}
\begin{lstlisting}
    const app = express();
\end{lstlisting}   
\vspace{1cm}

Cria uma instância do Express, que será usada para configurar e iniciar o servidor.

\vspace{1cm}
\begin{lstlisting}
    app.use(cors());
    app.use(express.json());
\end{lstlisting}
\vspace{1cm}

- cors(): Permite que o backend aceite requisições de outros domínios (útil em APIs públicas).

- express.json(): Middleware para processar o corpo das requisições no formato JSON.

\vspace{1cm}
\begin{lstlisting}
    mongoose.connect(process.env.MONGODB_URI)
      .then(() => console.log('Conectado ao MongoDB'))
      .catch(err => console.error('Erro ao conectar ao MongoDB:', err));
\end{lstlisting}
\vspace{1cm}

- mongoose.connect: Conecta o backend ao banco de dados MongoDB, 
usando a URI fornecida na variável de ambiente $MONGODB_URI$ do arquivo .env.

- .then() e .catch(): Gerenciam o sucesso ou falha da conexão.

\vspace{1cm}
\begin{lstlisting}
    app.use('/api/auth', authRoutes);
    app.use('/api/lojas', lojaRoutes);
    app.use('/api/users', userRoutes);
\end{lstlisting}
\vspace{1cm}

Define as rotas da aplicação:

- /api/auth: Redireciona as requisições relacionadas à autenticação para o authRoutes.

- /api/lojas: Redireciona requisições relacionadas às lojas.

- /api/users: Redireciona requisições relacionadas aos usuários.

\vspace{1cm}
\begin{lstlisting}
    const PORT = process.env.PORT || 3000;
    app.listen(PORT, () => {
      console.log(`Servidor rodando na porta ${PORT}`);
    });
\end{lstlisting}
\vspace{1cm}

- PORT: Define a porta do servidor, priorizando a porta especificada na variável de ambiente PORT. 
Caso não esteja definida, usa a porta 3000 como padrão.

- app.listen: Inicia o servidor e exibe uma mensagem de sucesso no console.

\subsection{Modelo da coleção Loja}

Vamos criar o modelo de dados para a coleção de lojas.

\vspace{1cm}
\begin{lstlisting}
    const mongoose = require('mongoose');
\end{lstlisting}  
\vspace{1cm}

- mongoose: Biblioteca para trabalhar com MongoDB em JavaScript, oferecendo um modelo orientado a objetos.

\vspace{1cm}
\begin{lstlisting}
        const lojaSchema = new mongoose.Schema({
    nome: {
        type: String,
        required: true
    },
    categoria: {
        type: String,
        required: true
    },
    localizacao: {
        type: String,
        required: true
    },
    descricao: String,
    programas: [{
        nome: String,
        progresso: Number,
        maxCarimbos: Number
    }],
    banner: String,
    logo: String,
    horario: String,
    proprietario: {
        type: mongoose.Schema.Types.ObjectId,
        ref: 'User'
    }
    }, {
    timestamps: true
    });
\end{lstlisting}  
\vspace{1cm}

\subsection{Controller da coleção Loja}

Vamos criar o controller para manipular as operações relacionadas às lojas no backend, 
usando o modelo Loja previamente definido. Esses controladores lidam com as ações de criação, 
consulta, atualização e manipulação de dados relacionados às lojas.

\vspace{1cm}
\begin{lstlisting}
        exports.criarLoja = async (req, res) => {
    try {
        const loja = new Loja({
        ...req.body,
        proprietario: req.user._id
        });
        await loja.save();
        res.status(201).json(loja);
    } catch (error) {
        res.status(400).json({ message: error.message });
    }
    };
\end{lstlisting}
\vspace{1cm}

Objetivo: Criar uma nova loja e armazená-la no banco de dados.
Como funciona:

- Recebe os dados da loja no corpo da requisição (req.body).

- Adiciona o ID do proprietário $(req.user._id)$ ao documento da loja. O proprietário deve ser o usuário autenticado.

- Salva a loja no banco de dados com loja.save().

- Retorna o documento criado em resposta com o código HTTP 201 (Criado).

- Em caso de erro, retorna um código 400 (Requisição inválida) com a mensagem do erro.

\vspace{1cm}
\begin{lstlisting}
        exports.buscarLojas = async (req, res) => {
    try {
        const lojas = await Loja.find();
        res.json(lojas);
    } catch (error) {
        res.status(500).json({ message: error.message });
    }
    };  
\end{lstlisting}
\vspace{1cm}

Objetivo: Buscar todas as lojas cadastradas.
Como funciona:

- Usa Loja.find() para recuperar todas as lojas do banco de dados.

- Retorna as lojas em formato JSON.

- Em caso de erro, retorna um código 500 (Erro interno do servidor) com a mensagem do erro.

\vspace{1cm}
\begin{lstlisting}
        exports.buscarLojaPorId = async (req, res) => {
    try {
        const loja = await Loja.findById(req.params.id);
        if (!loja) {
        return res.status(404).json({message: 'Loja não encontrada'});
        }
        res.json(loja);
    } catch (error) {
        res.status(500).json({ message: error.message });
    }
    };
\end{lstlisting}
\vspace{1cm}

Objetivo: Buscar uma loja específica pelo ID.
Como funciona:

- Usa Loja.findById(req.params.id) para recuperar a loja com o ID fornecido.

- Se a loja não for encontrada, retorna um código 404 (Não encontrado).

- Retorna a loja encontrada em formato JSON.

- Em caso de erro, retorna um código 500 (Erro interno do servidor) com a mensagem do erro.

\vspace{1cm}
\begin{lstlisting}
    exports.atualizarProgresso = async (req, res) => {
        try {
          const { lojaId, programaId, novoProgresso } = req.body;
          const loja = await Loja.findById(lojaId);
          
          if (!loja) {
            return res.status(404).json({ message: 'Loja não encontrada' });
          }
      
          const programa = loja.programas.id(programaId);
          if (!programa) {
            return res.status(404).json({ message: 'Programa não encontrado' });
          }
      
          programa.progresso = novoProgresso;
          await loja.save();
          
          res.json(programa);
        } catch (error) {
          res.status(400).json({ message: error.message });
        }
      };
\end{lstlisting}
\vspace{1cm}

Objetivo: Atualizar o progresso de um programa específico em uma loja.
Como funciona:

- Recebe o ID da loja, o ID do programa e o novo progresso no corpo da requisição (req.body).


- Usa Loja.findById(lojaId) para recuperar a loja com o ID fornecido.

- Verifica se a loja existe.

- Usa loja.programas.id(programaId) para encontrar o programa com o ID fornecido dentro da loja.

- Atualiza o progresso do programa com o novo valor fornecido.

- Salva a loja atualizada com loja.save().

- Retorna o programa atualizado em formato JSON.

- Em caso de erro, retorna um código 400 (Requisição inválida) com a mensagem do erro.

















